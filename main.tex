\documentclass{report}
\usepackage[utf8]{inputenc}
\DeclareUnicodeCharacter{2212}{-}
\usepackage{amsmath}
\usepackage{gensymb}
\usepackage[version=4]{mhchem}
\usepackage{fancyhdr}
\usepackage[margin=1.0in]{geometry}

\pagestyle{fancy}
\headheight=30pt
\fancyhf{}

\lhead{}
\rhead{Year 7-9 IGCSE Science by Show De Lin}
\thispagestyle{fancy}

\begin{document}

\chapter*{List of Physics Equations}
Suitable for Year 7 to Year 9 IGCSE Science students. SI basic or SI-derived units are bolded. 

\medskip

\begin{flushleft}

\section*{1. MECHANICS}

\section*{Speed}
\normalfont Speed is a measure of how fast an object moves. The unit of speed is \bf{meters per second (m/s)}.

\begin{equation}
Instantaneous \hspace{1mm} speed = \frac{distance}{short \hspace{1mm} time \hspace{1mm} interval}
\end{equation}

\begin{equation}
Average \hspace{1mm} speed = \frac{total \hspace{1mm}  distance}{long \hspace{1mm} time \hspace{1mm} interval}
\end{equation}

\section*{Volume}
\subsubsection*{Volume of regular solids}
\normalfont The unit of solid volume is \bf{cubic meter (m^3)}.

\begin{equation}
Volume = length \times width \times height
\end{equation} 

\subsubsection*{Volume of liquid}
\normalfont Measure the volume using a measuring cylinder. Units of liquid volume can be $cm^3$ or $ml$.

\subsubsection*{Volume of irregular solids} 
\normalfont Submerge an irregular solid object into a measuring cylinder with water. The increase in water volume is the volume of the object. Units of volume can be $cm^3$ or $ml$.

\vspace{3mm} 

\begin{equation}
Volume = volume(final) - volume(initial)
\end{equation}

\section*{Density}
\normalfont The unit of density is \bf{kilograms per cubic meter (kg/m^3)}. 

\begin{equation}
Density = \frac{mass}{volume}
\end{equation}

\section*{Force}
\normalfont Force, F is a push or pull. Force can be measured using a Newtonmeter, forcemeter or weighing balance. The unit of force is \bf{Newton (N)}. 

\section*{Weight}
\normalfont Weight, W is the force of gravity on an object. Weight can be measured with a spring balance. The unit of weight is \bf{Newton (N)}.

\section*{Mass}
\normalfont Mass, m is the amount of matter in an object. Mass can be measured with a mass balance. The unit of mass is \bf{kilogram (kg)}. 
\section*{Gravitational Constant}
\normalfont Everything on the surface of Earth experiences a constant pull of gravity. The gravitational constant, g on the Earth is \bf{10N/kg}.

\smallskip $g$ = $10 N/kg$

\begin{equation}
Weight = mass \times gravitational \hspace{1mm} constant 
\end{equation}

\section*{Pressure}
\normalfont Pressure of gas can be measured using a barometer. Pressure of liquid can be measured using a liquid manometer (U-shaped glass tube). The unit of pressure is \bf{pascal (Pa)}. 

\smallskip $1 Pa$ = $1N/m^2$

\begin{equation}
Pressure = \frac{force}{area}
\end{equation}

\section*{Moments}
\subsubsection*{Moment of a force}
\normalfont Moment measures the turning effect of a force around a pivot. The unit of moment is \bf{Nm}.

\begin{equation}
Moment = force \times perpendicular \hspace{1mm} distance \hspace{1mm} from \hspace{1mm} pivot
\end{equation}

\subsubsection{Principle of moments}
\normalfont A balanced beam in equilibrium will have equal amounts of clockwise and anticlockwise moments. 

\begin{equation}
clockwise \hspace{1mm } moment = anticlockwise \hspace{1mm} moment
\end{equation}

\begin{equation}
force_1 \times distance_1 = force_2 \times distance_2
\end{equation}

\newpage

\section*{2. ENERGY}
\normalfont The unit of energy is \bf{joule (J)}. 

\section*{Heat Energy}
\normalfont Heat energy is the energy spreading out from a hot object to the surroundings. 

\section*{Thermal Energy}
\normalfont Thermal energy is the heat energy stored in a hot object where molecules are constantly moving. The faster the molecules move, the higher the temperature. 

\section*{Temperature}
\normalfont Temperature is a degree of hotness. Temperature can be measured using a thermometer. The unit of temperature is degree Celsius (\degree C) or \bf{Kelvin (K)}. 

\begin{equation}
Temperature (K) = Temperature (\degree C) + 273.15
\end{equation}

\section*{Kinetic Energy}
\normalfont Kinetic energy, $E_k$ is the energy of a moving object.

\section*{Gravitational Potential Energy}
\normalfont Gravitational potential energy, $E_p$ is the energy stored by an object lifted to a height. 

\section*{Elastic Potential Energy}
\normalfont Elastic potential energy, $E_p$ is the energy stored in an object that is stretched, squashed or deformed.

\section*{Principle of Conservation of Energy}
\normalfont Energy cannot be created nor destroyed in an isolated system. The sum of kinetic energy, $E_k$ and potential energy, $E_p$ in the system must be constant.

\begin{equation}
E_k + E_p = constant
\end{equation}

\begin{equation}
E_k (initial) + E_p (initial) = E_k (final) + E_p (final)
\end{equation}

\end{flushleft}

\newpage

\section*{3. LIGHT}

\section*{Law of Reflection}
Law of reflection states that the angle of incidence, $\theta_i$ equals the angle of reflection, $\theta_r$. Angles are measured in \bf{degrees (\degree)}. 

\begin{equation}
Angle \hspace{1mm} of \hspace{1mm} incidence = Angle \hspace{1mm} of \hspace{1mm} reflection
\end{equation}

\centering {$\theta_i = \theta_r$}

\begin{flushleft}

\subsection*{Law of Refraction}
\normalfont Refraction is the bending of a light ray as it travels across mediums of different densities. 

\subsubsection*{Light moves along the normal}

\normalfont If light travels into a another medium along the normal, it does not bend.

\subsubsection*{Light moves towards a denser medium}

\normalfont If light travels into a denser medium, it will bend towards the normal. The angle of incidence will be bigger than the angle of refracted ray: 
    
\begin{equation}
Angle \hspace{1mm} of \hspace{1mm} incidence > Angle \hspace{1mm} of \hspace{1mm} refraction 
\end{equation}

\centering {$\theta_i > \theta_r$}

\end{flushleft}
\begin{flushleft}

\subsubsection*{Light moves towards a less dense medium}

\normalfont If light travels into a less dense medium, it will bend away from the normal. The angle of incidence will be smaller than the angle of refracted ray: 
    
\begin{equation}
Angle \hspace{1mm} of \hspace{1mm} incidence < Angle \hspace{1mm} of \hspace{1mm} refraction  
\end{equation}

\centering {$\theta_i < \theta_r$}

\end{flushleft}
\begin{flushleft}

\section*{Frequency}
\normalfont Frequency is the number of vibrations per second. The unit of frequency is \bf{Hertz (Hz)}.

\smallskip

\normalfont 1Hz = 1 vibration per second (${s^{-1}}$).

\begin{equation}
Frequency = \frac{number \hspace{1mm} of \hspace{1mm} vibrations}{time}
\end{equation}

\newpage

\section*{4. ELECTRICITY}
\normalfont At this level, you need not use these equations for calculations of electric current, voltage, resistance. However, understanding their definitions and Ohm's Law will be greatly beneficial to building your fundamentals. 

\section*{Current}
\normalfont Current can be measured using an ammeter. The unit of current is \bf{ampere (A)}. \normalfont The unit of charge is Coulomb (C). 

\medskip

\normalfont 1 ampere = 1 Coulomb of charge per second. 

\smallskip $1A$ = $1C/s$

\begin{equation}
    Current = \frac{charge \hspace{1mm} flow}{time}
\end{equation}

\section*{Voltage}
\normalfont Voltage can be measured using a voltmeter. The unit of voltage is \bf{volt (V)}. \normalfont The unit of energy is joules (J). The unit of charge is Coulomb (C). 

\medskip

\normalfont 1 volt = 1 joule of energy per Coulomb of charge. 

\smallskip $1V$ = $1J/C$

\begin{equation}
    Voltage = \frac{energy}{charge}
\end{equation}

\section*{Resistance}
\normalfont Resistance can be measured using an ohmmeter. The unit of resistance is \bf{ohm ($\Omega$)}. \normalfont The unit of potential difference is volt (V). The unit of current is ampere (A). 

\medskip

\normalfont 1 ohm = 1 volt of potential difference per ampere of current.

\smallskip $1\Omega$ = $1V/A$

\begin{equation}
    Resistance = \frac{potential \hspace{1mm} difference}{current}
\end{equation}

\section*{Ohm's Law}
\normalfont Ohm's Law states that the current, I passing through a conductor is directly proportional to the voltage, V and inversely proportional to the resistance, R. 

\begin{equation}
    Voltage = current \times resistance 
\end{equation}

\centering $V = I \cdot R$

\end{flushleft}

\end{document}

